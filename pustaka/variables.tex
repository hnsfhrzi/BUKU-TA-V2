% Atur variabel berikut sesuai namanya

% nama
\newcommand{\name}{Hans Fahrezi}
\newcommand{\authorname}{Fahrezi, Hans}
\newcommand{\nickname}{Hans}
\newcommand{\advisor}{Reza Fuad Rachmadi, S.T., M.T., Ph.D}
\newcommand{\coadvisor}{Dr. Supeno Mardi Susiki Nugroho, S.T., M.T.}
\newcommand{\examinerone}{ }
\newcommand{\examinertwo}{ }
\newcommand{\examinerthree}{ }
\newcommand{\headofdepartment}{Dr. Supeno Mardi Susiki Nugroho, S.T., M.T.}

% identitas
\newcommand{\nrp}{0721 19 4000 0043}
\newcommand{\advisornip}{19850403 201212 1 001}
\newcommand{\coadvisornip}{19700313 199512 1 001}
\newcommand{\examineronenip}{ }
\newcommand{\examinertwonip}{ }
\newcommand{\examinerthreenip}{ }
\newcommand{\headofdepartmentnip}{19700313 199512 1 001}

% judul
\newcommand{\tatitle}{RE-IDENTIFIKASI MOBIL PADA CITRA KAMERA PENGAWAS LALU LINTAS MENGGUNAKAN \emph{SWIN TRANSFORMER}}
\newcommand{\engtatitle}{CAR RE-IDENTIFICATION ON TRAFFIC MONITORING CAMERA IMAGES USING SWIN TRANSFORMER}

% tempat
\newcommand{\place}{Surabaya}

% jurusan
\newcommand{\studyprogram}{Teknik Komputer}
\newcommand{\engstudyprogram}{Computer Engineering}

% fakultas
\newcommand{\faculty}{Teknologi Elektro dan Informatika Cerdas}
\newcommand{\engfaculty}{Intelligent Electrical and Informatics Technology}

% singkatan fakultas
\newcommand{\facultyshort}{FTEIC}
\newcommand{\engfacultyshort}{FTEIC}

% departemen
\newcommand{\department}{Teknik Komputer}
\newcommand{\engdepartment}{Computer Engineering}

% kode mata kuliah
\newcommand{\coursecode}{EC224801}
