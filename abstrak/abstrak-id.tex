\begin{center}
  \large\textbf{ABSTRAK}
\end{center}

\addcontentsline{toc}{chapter}{ABSTRAK}

\vspace{2ex}

\begingroup
% Menghilangkan padding
\setlength{\tabcolsep}{0pt}

\noindent
\begin{tabularx}{\textwidth}{l >{\centering}m{2em} X}
  Nama Mahasiswa    & : & \name{}         \\

  Judul Tugas Akhir & : & \tatitle{}      \\

  Pembimbing        & : & 1. \advisor{}   \\
                    &   & 2. \coadvisor{} \\
\end{tabularx}
\endgroup

% Ubah paragraf berikut dengan abstrak dari tugas akhir
Indonesia memiliki skor kriminalitas yang cukup tinggi di dunia pada tahun 2021. \linebreak 
Berdasarkan data Criminal Scores yang dibuat oleh Global Initiative Against Transnational 
Organized Crime (GITOC), Indonesia tercatat memiliki skor 6.38 dan berada di urutan 25 dari 193
negara yang terdaftar di PBB. Pada tahun 2020, angka kejahatan yang terjadi di Indonesia sebanyak 
247.218 kasus dengan Sumatera Utara memiliki 32.990 kasus kejahatan. Beberapa kasus kejahatan 
yang terjadi sering menggunakan kendaraan khususnya mobil sebagai alat untuk
membantu pelaku dalam melaksanakan tindakannya. Dengan berkembangnya teknologi pada
saat ini, melakukan pelacakan mobil pelaku dapat menggunakan sebuah model re-identifikasi.
Model re-identifikasi mobil ini dapat mengidentifikasi mobil pada sebuah tangkapan citra
berdasarkan data mobil yang dimaksud. Pada penelitian tugas akhir ini, dikembangkan sebuah
model re-identifikasi menggunakan metode terbaru yaitu Swin Transformer. Swin Transformer
merupakan sebuah arsitektur terbaru yang dikembangkan dari arsitektur Convolutional Neural
Network (CNN) dan Vision Transformer, sehingga memiliki kelebihan dibandingkan arsitektur
yang sudah lama ada. Dari pengembangan model reidentifikasi mobil dengan metode Swin
Transformer, diharapkan dapat mempermudah dalam pelacakan dan pemecahan kasus yang ditangani 
oleh polisi.

% Ubah kata-kata berikut dengan kata kunci dari tugas akhir
Kata Kunci: kriminalitas, re-identifikasi, \emph{Swin Transformer}, VRIC.
