\begin{center}
  \large\textbf{ABSTRACT}
\end{center}

\addcontentsline{toc}{chapter}{ABSTRACT}

\vspace{2ex}

\begingroup
% Menghilangkan padding
\setlength{\tabcolsep}{0pt}

\noindent
\begin{tabularx}{\textwidth}{l >{\centering}m{3em} X}
  \emph{Name}     & : & \name{}         \\

  \emph{Title}    & : & \engtatitle{}   \\

  \emph{Advisors} & : & 1. \advisor{}   \\
                  &   & 2. \coadvisor{} \\
\end{tabularx}
\endgroup

% Ubah paragraf berikut dengan abstrak dari tugas akhir dalam Bahasa Inggris
\emph{Indonesia has a high crime score in the world in 2021. Based on Criminal Scores data
created by Global Initiative Against Transnational Organized Crime (GITOC), Indonesia is
recorded as having a score of 6.38 and ranked 25th out of 193 countries that registered at United
Nations (UN). In 2020, number of crimes that occurred in Indonesia was 247.218 cases with
North Sumatera having 32.990 crime cases. Some crime cases that occur often use vehicles,
especially cars, as a tool to assist perpetrator in carrying out their actions. With the development 
of technology at this time, tracking the perpetrator's car can use a re-identification model.
This car re-identification model can identify cars in an image capture based on the car's data.
In this final project research, a re-identification model was developed using the latest method,
and that's Swin Transformer. Swin Transformer is the latest architecture developed from Convolutional 
Neural Network (CNN) and Vision Transformer architectures, so it has advantages
over the old architectures. From this development of the car re-identification model using Swin
Transformer, hopes that it will make it easier to track and solve cases handled by the police.}

% Ubah kata-kata berikut dengan kata kunci dari tugas akhir dalam Bahasa Inggris
\emph{Keywords}: \emph{Criminality}, \emph{Re-identification}, \emph{Swin Transformer}, \emph{VRIC}.
