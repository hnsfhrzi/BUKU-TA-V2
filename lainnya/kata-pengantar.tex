\begin{center}
  \Large
  \textbf{KATA PENGANTAR}
\end{center}

\addcontentsline{toc}{chapter}{KATA PENGANTAR}

\vspace{2ex}

% Ubah paragraf-paragraf berikut dengan isi dari kata pengantar

Puji dan syukur kehadirat Tuhan yang Maha Esa yang telah memberikan Rahmat serta Karunia-nya 
dan atas segala limpahan berkah, sehingga penulis dapat menyelesaikan penelitian ini dengan judul 
"Re-identifikasi Mobil Pada Citra Kamera Pengawas Lalu Lintas Menggunakan Swin Transformer"

Penelitian ini disusun dalam rangka pemenuhan bidang riset beserta persyaratan menyelesaikan 
pendidikan S1 di Departemen Teknik Komputer, Institut Teknologi Sepuluh Nopember Surabaya.

Penulis menyadari bawasannya penelitian ini masih belum sempurna karena keterbatasan kemampuan 
dan pengetahuan penulis. Oleh karena itu penulis dengan rendah hati mohon maaf atas segala 
kekurangan yang ada. Penulis juga menyadari penyusunan penelitian ini mendapat banyak 
bantuan, bimbingan, dan dukungan dari berbagai pihak sehingga penelitian ini dapat diselesaikan

Oleh karena itu, penulis mengucapkan terima kasih kepada:

\begin{enumerate}[nolistsep]

  \item Keluarga, Ibu, Bapak dan Saudari tercinta yang selalu memberikan dukungan spiritual, moral, maupun 
  material dalam penyusunan penelitian ini.

  \item Bapak Dr. Supeno Mardi Susiki Nugroho, ST., MT., selaku Kepala Departemen Teknik Komputer, Fakultas 
  Teknologi Elektro dan Informatika Cerdas, Institut Teknologi Sepuluh Nopember Surabaya.

  \item Bapak Reza Fuad Rachmadi, ST., MT., Ph.D., selaku dosen pembimbing I dan Bapak Dr. Supeno Mardi 
  Susiki Nugroho, ST., MT., selaku dosen pembimbing II yang selalu memberikan arahan, bimbingan, dan semangat 
  selama mengerjakan penelitian Tugas Akhir ini.

  \item Bapak-ibu dosen pengajar Departemen Teknik Komputer, atas pengajaran, bimbingan, serta perhatian 
  yang diberikan kepada penulis selama ini.

  \item Seluruh sahabat dan teman-teman dari Teknik Komputer angkatan 2019.

\end{enumerate}

Akhir kata, kesempurnaan hanya milik Tuhan, untuk itu penulis memohon segenap kritik dan saran 
yang membangun. Semoga penelitian ini dapat memberikan manfaat bagi kita semua, amin.

\begin{flushright}
  \begin{tabular}[b]{c}
    \place{}, \MONTH{} \the\year{} \\
    \\
    \\
    \\
    \\
    \name{}
  \end{tabular}
\end{flushright}
