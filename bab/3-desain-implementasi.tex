\chapter{METODOLOGI}
\label{chap:metodologi}

% Ubah bagian-bagian berikut dengan isi dari desain dan implementasi

% Penelitian ini dilaksanakan sesuai dengan batasan-batasan masalah yang telah disebutkan
% pada bagian 1.3. Berikut ini adalah beberapa peralatan dan perangkat yang terlibat dalam
% penelitian ini.

\section{Metode yang Digunakan}
\label{sec:metodeyangdigunakan}

Tugas akhir ini merupakan penelitian di bidang visi komputer dengan tujuan untuk melakukan 
re-identifikasi mobil menggunakan kumpulan citra dari dataset VRIC. Alur blok diagram metodologi 
dapat dilihat pada gambar 3.1. Perancangan model menggunakan metode Swin Transformer dari 
\emph{Pytorch ReID layumi}.

\begin{figure}[ht]
  \centering
  % Nama dari file gambar yang diinputkan
  \includegraphics[scale=0.65]{gambar/metodologiFix.jpg}
  % Keterangan gambar yang diinputkan
  \caption{Diagram Blok Metodologi}
  % Label referensi dari gambar yang diinputkan
  \label{fig:diagramblokmetodologi}
\end{figure}

\begin{enumerate}[nolistsep]

  \item \textbf{Pengolahan Dataset VRIC}
  
  Pada tahapan ini, dilakukan proses pengolahan dataset yang didapat dengan 
  membagi dataset menjadi dua yaitu data training dan data test. Kedua data 
  ini nantinya akan digunakan untuk tujuan yang berbeda.

  \item \textbf{Pelatihan Model Swin Transformer}
  
  Pada tahapan ini, dilakukan training dan tuning pada model Swin Transformer 
  dengan mengubah parameternya. Parameter yang perlu disetel seperti 
  \emph{Loss Function},\emph{Optimizer Function}, dan \emph{Epoch}.

  \item \textbf{Pengujian Model}
  
  Pada tahapan ini, model Swin Transformer yang telah melalui training dan 
  tuning akan diuji. Pengujian dilakukan menggunakan data test VRIC yang sebelumnya 
  telah dibagi.

  \item \textbf{Hasil}
  
  Jika hasil dari pengujian telah mencapai threshold dan model dapat melakukan \linebreak
  re-identifikasi dengan baik, maka seluruh hasil pengujian dicatat untuk dibandingkan 
  dengan penelitian yang sudah ada sebelumnya.

\end{enumerate}

\section{Bahan dan Peralatan yang Digunakan}
\label{sec:bahandanperalatanyangdigunakan}

\subsection{Dataset}

Dataset yang digunakan dalam penelitian ini adalah VRIC dataset. VRIC dataset memuat 
60 ribu citra mobil yang diambil dari kamera CCTV dari berbagai sudut pandang, sudut 
pencahayaan, dan kondisi lingkungan yang bermacam-macam. Setiap citra mobil dari VRIC 
dataset telah diberikan label sesuai dengan id vehicle dan id kamera yang mengambil 
citra tersebut.

%gambar 3.2 dataset

Gambar 3.2 merupakan beberapa contoh citra yang ada di VRIC dataset. Di dalam VRIC 
dataset, 60 ribu citra mobil telah dibagi ke tiga folder yang berbeda, yaitu folder 
\emph{train}, \emph{gallery}, dan \emph{query}. Folder \emph{train} berisikan 54 
ribu citra, sementara pada folder \emph{gallery} dan \emph{query} masing-masing 
berisikan 2 ribu citra. Terdapat pula list anotasi dari seluruh citra yang berisikan 
format nama file citra, label ID dari mobil, dan label kamera.

\subsection{Laptop}

Laptop digunakan untuk melakukan pemisahan (\emph{splitting}) dataset, training model, 
dan melakukan testing model yang telah selesai ditraining menggunakan VRIC dataset. 
Laptop perlu terhubung dengan internet untuk mengakses google Colaboratory, yang 
merupakan environment yang dibutuhkan untuk membuat model Swin Transformer. Spesifikasi 
laptop yang digunakan dapat dilihat pada tabel 3.1.

\begin{table}[ht]
  \begin{center}
  \caption{Spesifikasi Perangkat Laptop yang Digunakan}
  \label{tb:spesifikasilaptop}
  \begin{tabular}{|l|l|}
      \hline
      \textit{\textbf{Processor}}        & \begin{tabular}[c]{@{}l@{}}Intel Core i7-8750H \\ CPU @ 2.2 GHz\end{tabular} \\ \hline
      \textit{\textbf{Storage}}          & \begin{tabular}[c]{@{}l@{}}SSHD 1 TB Storage\\ SSD 128 GB Storage\end{tabular}         \\ \hline
      \textit{\textbf{RAM}}              & \begin{tabular}[c]{@{}l@{}}16 GB SODIMM DDR4 \\ 2666 MHz Dual Channel\end{tabular}    \\ \hline
      \textit{\textbf{Graphic Card}}     & \begin{tabular}[c]{@{}l@{}}NVIDIA GeForce GTX 1050 Ti \\ 4 GB GDDR5\end{tabular}      \\ \hline
      \textit{\textbf{Operating System}} & \begin{tabular}[c]{@{}l@{}}Windows 10 Home \\ Single Language 64-bit\end{tabular}     \\ \hline
  \end{tabular}
  \end{center}
\end{table}

\subsection{Google Colaboratory}

Google Colaboratory merupakan sebuah environment komputasi cloud dengan format \linebreak notebook 
yang disediakan oleh \emph{Google research} dengan fungsi untuk menjalankan kode python 
yang telah dituliskan. Google Colaboratory sangat membantu dalam pengerjaan \emph{data science}, 
\emph{deep learning}, dan \emph{machine learning} dikarenakan spesifikasi \emph{hardware} 
yang tinggi dari Google Colaboratory. Spesifikasi Google Colaboratory yang digunakan dapat 
dilihat pada tabel 3.2.

\begin{table}[ht]
  \begin{center}
  \caption{Spesifikasi Google Colaboratory}
  \label{tb:spesifikasigooglecolaboratory}
  \begin{tabular}{|l|l|}
      \hline
      \textit{\textbf{Processor}}        & \begin{tabular}[c]{@{}l@{}}Intel(R) Xeon(R) \\ CPU @ 2.2 GHz\end{tabular} \\ \hline
      \textit{\textbf{Graphic Card}}     & \begin{tabular}[c]{@{}l@{}}NVIDIA A100 SXM \\ 40 GB HBM2\end{tabular}      \\ \hline
  \end{tabular}
  \end{center}
\end{table}

\section{Urutan Pelaksanaan Penelitian}
\label{sec:urutanpelaksanaanpenelitian}

\subsection{Pemisahan Dataset}

VRIC dataset terbagi menjadi 3 bagian, yaitu data training, data query, dan data gallery. Ketiga 
bagian dataset tersebut telah dipisahkan dan disesuaikan secara langsung oleh developer dari VRIC 
dataset untuk kegunaan re-identifikasi mobil. Selain file berisi dataset, di dalam VRIC dataset 
juga terdapat file txt yang berisi list anotasi dari seluruh citra, dengan format nama file citra, 
label ID dari mobil, dan label kamera. Dengan menggunakan list anotasi tersebut, maka setiap dataset 
perlu dilakukan perubahan nama file sesuai dengan label ID dan label kameranya.

% Potongan kode rename
\lstinputlisting[
  language=Python,
  caption={Program Rename Dataset.},
  label={lst:renamedataset}
]{program/perubahan-nama-file.py}

Pada penelitian ini, diperlukan satu bagian lagi dalam percobaannya, yaitu data validasi. Data validasi 
berfungsi untuk melakukan validasi dan mengecek setiap epoch yang telah dilakukan pada training. Data 
validasi dibuat dengan mengambil sebagian data training untuk dipindahkan ke data validasi.

% Potongan kode splitting
\lstinputlisting[
  language=Python,
  caption={Program Pembagian Dataset.},
  label={lst:splittingdataset}
]{program/splitting-dataset.py}

\subsection{\emph{Training} dan \emph{Validation}}

\subsection{Model}

\subsection{Pengujian Model}

% Alat diimplementasikan dengan \lipsum[1]

% % Contoh pembuatan potongan kode
% \begin{lstlisting}[
%   language=C++,
%   caption={Program halo dunia.},
%   label={lst:halodunia}
% ]
% #include <iostream>

% int main() {
%     std::cout << "Halo Dunia!";
%     return 0;
% }
% \end{lstlisting}

% \lipsum[2-3]

% % Contoh input potongan kode dari file
% \lstinputlisting[
%   language=Python,
%   caption={Program perhitungan bilangan prima.},
%   label={lst:bilanganprima}
% ]{program/bilangan-prima.py}

% \lipsum[4]
