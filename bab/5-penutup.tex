\chapter{PENUTUP}
\label{chap:penutup}

% Ubah bagian-bagian berikut dengan isi dari penutup

\section{Kesimpulan}
\label{sec:kesimpulan}

Berdasarkan hasil pengujian sementara yang telah dijelaskan pada Bab 4, dapat 
ditarik beberapa kesimpulan sebagai berikut:

\begin{enumerate}[nolistsep]

  \item Pembuatan Model Re-Identifikasi dengan metode Swin Transformer untuk 
  mengidentifikasi ulang mobil berhasil dibuat dengan nilai mAP paling optimal sampai 
  saat ini yang didapat sebesar 72,5\%.

  \item Dari keseluruhan varian model yang telah dibuat pada penelitian sampai saat ini, 
  model terbaik yang dapat digunakan untuk model re-identifikasi mobil adalah 
  Swin Transformer V2 dengan pengaturan parameter 1. Hal ini dibuktikan dengan nilai 
  mAP yang didapat sebesar 72,5\% dan nilai rank@1 sebesar 68,6\%.

  % \item \lipsum[2][7-10]

\end{enumerate}

\section{Saran}
\label{chap:saran}

Untuk pengembangan lebih lanjut pada penelitian berikutnya atau yang akan datang, \linebreak
penulis memiliki beberapa saran antara lain:

% \begin{enumerate}[nolistsep]

%   % \item Menggunakan Dataset citra yang berasal dari daerah lokal atau seminimalnya berasal 
%   % dari Indonesia.

%   % % \item \lipsum[2][4-6]

%   % % \item \lipsum[2][7-10]

% \end{enumerate}
