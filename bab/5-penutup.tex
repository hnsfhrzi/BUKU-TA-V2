\chapter{PENUTUP}
\label{chap:penutup}

% Ubah bagian-bagian berikut dengan isi dari penutup

\section{Kesimpulan}
\label{sec:kesimpulan}

Berdasarkan hasil pengujian yang telah dijelaskan pada Bab 4, dapat 
ditarik beberapa kesimpulan sebagai berikut:

\begin{enumerate}[nolistsep]

  \item Pembuatan Model Re-Identifikasi dengan metode Swin Transformer untuk 
  mengidentifikasi ulang mobil berhasil dibuat dengan nilai mAP tertinggi yang didapat 
  di penelitian ini sebesar 75,26\% dan rank@1 tertinggi sebesar 71,22\%.

  \item Dari keseluruhan model yang telah dibuat pada penelitian ini, 
  model terbaik yang dapat digunakan untuk model re-identifikasi mobil adalah 
  Swin Transformer V2 dengan pengaturan parameter 2. Hal ini dibuktikan dengan nilai 
  mAP rata-rata yang didapat sebesar 75,23\% dan rank@1 rata-rata sebesar 71,2\%.

  % \item Arsitektur Swin Transformer cocok digunakan untuk menyelesaikan permasalahan 
  % re-identifikasi kendaraan karena dapat mengidentifikasi citra dengan resolusi yang cukup 
  % kecil, sehingga melakukan re-identifikasi dengan sebuah citra dengan objek yang 
  % tidak terlalu jelas dapat dilakukan.

  \item Model re-identifikasi mobil yang dibuat di penelitian ini masih belum mengungguli 
  model re-identifikasi dari penelitian terdahulu yang menggunakan VRIC sebagai dataset 
  dan CNN sebagai arsitektur.

\end{enumerate}

\section{Saran}
\label{chap:saran}

Untuk pengembangan lebih lanjut pada penelitian berikutnya atau yang akan datang, \linebreak
penulis memiliki beberapa saran antara lain:

\begin{enumerate}[nolistsep]

  \item Menggunakan Dataset dengan citra yang diambil dari kamera pengawas lalu lintas di 
  kota-kota di Indonesia.

  % \item Menggunakan Dataset citra yang berasal dari daerah lokal atau berasal dari Indonesia.

  \item Perlunya pembenahan dan pengaturan pada hyper-parameter agar model dapat melakukan 
  re-identifikasi dengan lebih baik.

  \item Mencoba menggunakan varian model lain dari Swin Transformer di penelitian kedepannya.

\end{enumerate}
