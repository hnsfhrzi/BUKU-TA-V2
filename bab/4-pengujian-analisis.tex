\chapter{HASIL DAN PEMBAHASAN}
\label{chap:hasildanpembahasan}

% Ubah bagian-bagian berikut dengan isi dari pengujian dan analisis

Pada penelitian ini dipaparkan hasil penelitian beserta analisis dari model re-identifikasi yang telah dibuat sesuai dengan 
yang telah dijelaskan di Bab 3. Model re-identifikasi yang telah dibuat akan diuji cobakan menggunakan data test dari dataset 
VRIC yang telah disiapkan. 

\section{Hasil Penelitian}
\label{sec:hasilpenelitian}

Pada penelitian ini, telah ditentukan 4 jenis model re-identifikasi yang berbeda berdasarkan jenis model dan pengaturan 
hyper-parameternya. Dan pada setiap jenis model, diciptakan 3 model re-identifikasi untuk menentukan model re-identifikasi 
terbaik di setiap jenisnya. Sehingga total model yang diciptakan pada penelitian ini berjumlah 12 model. 

Dalam proses \emph{deep learning} khususnya pada tahap \emph{training} dan \emph{validation}, terdapat nilai yang dapat 
menunjukan bagus atau tidaknya model yang telah diciptakan. Nilai tersebut adalah nilai loss dan nilai top 1 error, yang dapat 
menunjukan performa dari sebuah model saat tahapan \emph{training} dan \emph{validation} berlangsung. 

\section{Analisa Hasil}
\label{sec:analisahasil}

Dari pengujian yang \lipsum[1]

% Contoh pembuatan tabel
\begin{longtable}{|c|c|c|}
  \caption{Hasil Pengukuran Energi dan Kecepatan}
  \label{tb:EnergiKecepatan}                                   \\
  \hline
  \rowcolor[HTML]{C0C0C0}
  \textbf{Energi} & \textbf{Jarak Tempuh} & \textbf{Kecepatan} \\
  \hline
  10 J            & 1000 M                & 200 M/s            \\
  20 J            & 2000 M                & 400 M/s            \\
  30 J            & 4000 M                & 800 M/s            \\
  40 J            & 8000 M                & 1600 M/s           \\
  \hline
\end{longtable}

\lipsum[2-4]
